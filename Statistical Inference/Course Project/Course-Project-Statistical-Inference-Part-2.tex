\documentclass[]{article}
\usepackage{lmodern}
\usepackage{amssymb,amsmath}
\usepackage{ifxetex,ifluatex}
\usepackage{fixltx2e} % provides \textsubscript
\ifnum 0\ifxetex 1\fi\ifluatex 1\fi=0 % if pdftex
  \usepackage[T1]{fontenc}
  \usepackage[utf8]{inputenc}
\else % if luatex or xelatex
  \ifxetex
    \usepackage{mathspec}
  \else
    \usepackage{fontspec}
  \fi
  \defaultfontfeatures{Ligatures=TeX,Scale=MatchLowercase}
\fi
% use upquote if available, for straight quotes in verbatim environments
\IfFileExists{upquote.sty}{\usepackage{upquote}}{}
% use microtype if available
\IfFileExists{microtype.sty}{%
\usepackage{microtype}
\UseMicrotypeSet[protrusion]{basicmath} % disable protrusion for tt fonts
}{}
\usepackage[margin=1in]{geometry}
\usepackage{hyperref}
\hypersetup{unicode=true,
            pdftitle={Statistical Inference Course Project Part 2},
            pdfauthor={Luis Fernando Agottani},
            pdfborder={0 0 0},
            breaklinks=true}
\urlstyle{same}  % don't use monospace font for urls
\usepackage{color}
\usepackage{fancyvrb}
\newcommand{\VerbBar}{|}
\newcommand{\VERB}{\Verb[commandchars=\\\{\}]}
\DefineVerbatimEnvironment{Highlighting}{Verbatim}{commandchars=\\\{\}}
% Add ',fontsize=\small' for more characters per line
\usepackage{framed}
\definecolor{shadecolor}{RGB}{248,248,248}
\newenvironment{Shaded}{\begin{snugshade}}{\end{snugshade}}
\newcommand{\AlertTok}[1]{\textcolor[rgb]{0.94,0.16,0.16}{#1}}
\newcommand{\AnnotationTok}[1]{\textcolor[rgb]{0.56,0.35,0.01}{\textbf{\textit{#1}}}}
\newcommand{\AttributeTok}[1]{\textcolor[rgb]{0.77,0.63,0.00}{#1}}
\newcommand{\BaseNTok}[1]{\textcolor[rgb]{0.00,0.00,0.81}{#1}}
\newcommand{\BuiltInTok}[1]{#1}
\newcommand{\CharTok}[1]{\textcolor[rgb]{0.31,0.60,0.02}{#1}}
\newcommand{\CommentTok}[1]{\textcolor[rgb]{0.56,0.35,0.01}{\textit{#1}}}
\newcommand{\CommentVarTok}[1]{\textcolor[rgb]{0.56,0.35,0.01}{\textbf{\textit{#1}}}}
\newcommand{\ConstantTok}[1]{\textcolor[rgb]{0.00,0.00,0.00}{#1}}
\newcommand{\ControlFlowTok}[1]{\textcolor[rgb]{0.13,0.29,0.53}{\textbf{#1}}}
\newcommand{\DataTypeTok}[1]{\textcolor[rgb]{0.13,0.29,0.53}{#1}}
\newcommand{\DecValTok}[1]{\textcolor[rgb]{0.00,0.00,0.81}{#1}}
\newcommand{\DocumentationTok}[1]{\textcolor[rgb]{0.56,0.35,0.01}{\textbf{\textit{#1}}}}
\newcommand{\ErrorTok}[1]{\textcolor[rgb]{0.64,0.00,0.00}{\textbf{#1}}}
\newcommand{\ExtensionTok}[1]{#1}
\newcommand{\FloatTok}[1]{\textcolor[rgb]{0.00,0.00,0.81}{#1}}
\newcommand{\FunctionTok}[1]{\textcolor[rgb]{0.00,0.00,0.00}{#1}}
\newcommand{\ImportTok}[1]{#1}
\newcommand{\InformationTok}[1]{\textcolor[rgb]{0.56,0.35,0.01}{\textbf{\textit{#1}}}}
\newcommand{\KeywordTok}[1]{\textcolor[rgb]{0.13,0.29,0.53}{\textbf{#1}}}
\newcommand{\NormalTok}[1]{#1}
\newcommand{\OperatorTok}[1]{\textcolor[rgb]{0.81,0.36,0.00}{\textbf{#1}}}
\newcommand{\OtherTok}[1]{\textcolor[rgb]{0.56,0.35,0.01}{#1}}
\newcommand{\PreprocessorTok}[1]{\textcolor[rgb]{0.56,0.35,0.01}{\textit{#1}}}
\newcommand{\RegionMarkerTok}[1]{#1}
\newcommand{\SpecialCharTok}[1]{\textcolor[rgb]{0.00,0.00,0.00}{#1}}
\newcommand{\SpecialStringTok}[1]{\textcolor[rgb]{0.31,0.60,0.02}{#1}}
\newcommand{\StringTok}[1]{\textcolor[rgb]{0.31,0.60,0.02}{#1}}
\newcommand{\VariableTok}[1]{\textcolor[rgb]{0.00,0.00,0.00}{#1}}
\newcommand{\VerbatimStringTok}[1]{\textcolor[rgb]{0.31,0.60,0.02}{#1}}
\newcommand{\WarningTok}[1]{\textcolor[rgb]{0.56,0.35,0.01}{\textbf{\textit{#1}}}}
\usepackage{graphicx,grffile}
\makeatletter
\def\maxwidth{\ifdim\Gin@nat@width>\linewidth\linewidth\else\Gin@nat@width\fi}
\def\maxheight{\ifdim\Gin@nat@height>\textheight\textheight\else\Gin@nat@height\fi}
\makeatother
% Scale images if necessary, so that they will not overflow the page
% margins by default, and it is still possible to overwrite the defaults
% using explicit options in \includegraphics[width, height, ...]{}
\setkeys{Gin}{width=\maxwidth,height=\maxheight,keepaspectratio}
\IfFileExists{parskip.sty}{%
\usepackage{parskip}
}{% else
\setlength{\parindent}{0pt}
\setlength{\parskip}{6pt plus 2pt minus 1pt}
}
\setlength{\emergencystretch}{3em}  % prevent overfull lines
\providecommand{\tightlist}{%
  \setlength{\itemsep}{0pt}\setlength{\parskip}{0pt}}
\setcounter{secnumdepth}{0}
% Redefines (sub)paragraphs to behave more like sections
\ifx\paragraph\undefined\else
\let\oldparagraph\paragraph
\renewcommand{\paragraph}[1]{\oldparagraph{#1}\mbox{}}
\fi
\ifx\subparagraph\undefined\else
\let\oldsubparagraph\subparagraph
\renewcommand{\subparagraph}[1]{\oldsubparagraph{#1}\mbox{}}
\fi

%%% Use protect on footnotes to avoid problems with footnotes in titles
\let\rmarkdownfootnote\footnote%
\def\footnote{\protect\rmarkdownfootnote}

%%% Change title format to be more compact
\usepackage{titling}

% Create subtitle command for use in maketitle
\providecommand{\subtitle}[1]{
  \posttitle{
    \begin{center}\large#1\end{center}
    }
}

\setlength{\droptitle}{-2em}

  \title{Statistical Inference Course Project Part 2}
    \pretitle{\vspace{\droptitle}\centering\huge}
  \posttitle{\par}
    \author{Luis Fernando Agottani}
    \preauthor{\centering\large\emph}
  \postauthor{\par}
      \predate{\centering\large\emph}
  \postdate{\par}
    \date{24/08/2019}


\begin{document}
\maketitle

For the second part, the ToothGrowth data in the R dataset package will
be analyzed.

\hypertarget{part-2-basic-inferential-data-analysis-instructions}{%
\subsection{Part 2: Basic Inferential Data Analysis
Instructions}\label{part-2-basic-inferential-data-analysis-instructions}}

\begin{enumerate}
\def\labelenumi{\arabic{enumi}.}
\tightlist
\item
  Load the ToothGrowth data and perform some basic exploratory data
  analyses
\end{enumerate}

\begin{Shaded}
\begin{Highlighting}[]
\KeywordTok{library}\NormalTok{(datasets)}

\KeywordTok{data}\NormalTok{(ToothGrowth)}
\end{Highlighting}
\end{Shaded}

\begin{enumerate}
\def\labelenumi{\arabic{enumi}.}
\setcounter{enumi}{1}
\tightlist
\item
  Provide a basic summary of the data.
\end{enumerate}

\begin{Shaded}
\begin{Highlighting}[]
\KeywordTok{dim}\NormalTok{(ToothGrowth)}
\end{Highlighting}
\end{Shaded}

\begin{verbatim}
## [1] 60  3
\end{verbatim}

\begin{Shaded}
\begin{Highlighting}[]
\KeywordTok{str}\NormalTok{(ToothGrowth)}
\end{Highlighting}
\end{Shaded}

\begin{verbatim}
## 'data.frame':    60 obs. of  3 variables:
##  $ len : num  4.2 11.5 7.3 5.8 6.4 10 11.2 11.2 5.2 7 ...
##  $ supp: Factor w/ 2 levels "OJ","VC": 2 2 2 2 2 2 2 2 2 2 ...
##  $ dose: num  0.5 0.5 0.5 0.5 0.5 0.5 0.5 0.5 0.5 0.5 ...
\end{verbatim}

\begin{Shaded}
\begin{Highlighting}[]
\KeywordTok{head}\NormalTok{(ToothGrowth)}
\end{Highlighting}
\end{Shaded}

\begin{verbatim}
##    len supp dose
## 1  4.2   VC  0.5
## 2 11.5   VC  0.5
## 3  7.3   VC  0.5
## 4  5.8   VC  0.5
## 5  6.4   VC  0.5
## 6 10.0   VC  0.5
\end{verbatim}

\begin{Shaded}
\begin{Highlighting}[]
\KeywordTok{summary}\NormalTok{(ToothGrowth)}
\end{Highlighting}
\end{Shaded}

\begin{verbatim}
##       len        supp         dose      
##  Min.   : 4.20   OJ:30   Min.   :0.500  
##  1st Qu.:13.07   VC:30   1st Qu.:0.500  
##  Median :19.25           Median :1.000  
##  Mean   :18.81           Mean   :1.167  
##  3rd Qu.:25.27           3rd Qu.:2.000  
##  Max.   :33.90           Max.   :2.000
\end{verbatim}

The data ToothGrowth give to us results from testing vitamin C in Guinea
Pigs to measure the tooth growth in differents dosages (0,5; 1,0; 2,0
mg) and in differents types of supplements (VC and OJ).

\begin{Shaded}
\begin{Highlighting}[]
\KeywordTok{library}\NormalTok{(ggplot2)}

\KeywordTok{qplot}\NormalTok{(supp,len,}\DataTypeTok{data=}\NormalTok{ToothGrowth, }\DataTypeTok{facets=}\OperatorTok{~}\NormalTok{dose, }\DataTypeTok{main=}\StringTok{"Tooth growth of guinea pigs by supplement type and dosage (mg)"}\NormalTok{,}\DataTypeTok{xlab=}\StringTok{"Supplement type"}\NormalTok{, }\DataTypeTok{ylab=}\StringTok{"Tooth length"}\NormalTok{) }\OperatorTok{+}\StringTok{ }\KeywordTok{geom_boxplot}\NormalTok{(}\KeywordTok{aes}\NormalTok{(}\DataTypeTok{fill =}\NormalTok{ supp))}
\end{Highlighting}
\end{Shaded}

\includegraphics{Course-Project-Statistical-Inference-Part-2_files/figure-latex/Exploratory Analisis-1.pdf}

The result of the exploratory analises from the data ``ToothGrowth'',
show that the tooth length increases when the dosage of vitamin C
bigger, and for dosages of 0,5mg and 1mg, the supplement OJ has better
results when comparing to VC. Lets confirm these with hypothesis.

\begin{enumerate}
\def\labelenumi{\arabic{enumi}.}
\setcounter{enumi}{2}
\tightlist
\item
  Use confidence intervals and/or hypothesis tests to compare tooth
  growth by supp and dose. (Only use the techniques from class, even if
  there's other approaches worth considering)
\end{enumerate}

Hypothesis to the supplement.

Lets work with Hypothesis. Using our null hypothesis when using
supplement OJ and VC.

So, Null Ho = lenth OJ = lenght VC, Alternative Ha = lenth OJ
\textgreater{} lenght VC.

\begin{Shaded}
\begin{Highlighting}[]
\NormalTok{OJ =}\StringTok{ }\NormalTok{ToothGrowth}\OperatorTok{$}\NormalTok{len[ToothGrowth}\OperatorTok{$}\NormalTok{supp }\OperatorTok{==}\StringTok{ 'OJ'}\NormalTok{]}
\NormalTok{VC =}\StringTok{ }\NormalTok{ToothGrowth}\OperatorTok{$}\NormalTok{len[ToothGrowth}\OperatorTok{$}\NormalTok{supp }\OperatorTok{==}\StringTok{ 'VC'}\NormalTok{]}

\KeywordTok{t.test}\NormalTok{(OJ, VC, }\DataTypeTok{alternative =} \StringTok{"greater"}\NormalTok{, }\DataTypeTok{paired =} \OtherTok{FALSE}\NormalTok{, }\DataTypeTok{var.equal =} \OtherTok{FALSE}\NormalTok{, }\DataTypeTok{conf.level =} \FloatTok{0.95}\NormalTok{)}
\end{Highlighting}
\end{Shaded}

\begin{verbatim}
## 
##  Welch Two Sample t-test
## 
## data:  OJ and VC
## t = 1.9153, df = 55.309, p-value = 0.03032
## alternative hypothesis: true difference in means is greater than 0
## 95 percent confidence interval:
##  0.4682687       Inf
## sample estimates:
## mean of x mean of y 
##  20.66333  16.96333
\end{verbatim}

With the graphic, we can see that for dosage equal 2mg, there is not a
significant difference. Lets check.

\begin{Shaded}
\begin{Highlighting}[]
\NormalTok{OJ2mg =}\StringTok{ }\NormalTok{ToothGrowth}\OperatorTok{$}\NormalTok{len[ToothGrowth}\OperatorTok{$}\NormalTok{supp }\OperatorTok{==}\StringTok{ 'OJ'} \OperatorTok{&}\StringTok{ }\NormalTok{ToothGrowth}\OperatorTok{$}\NormalTok{dose }\OperatorTok{==}\StringTok{ }\DecValTok{2}\NormalTok{]}
\NormalTok{VC2mg =}\StringTok{ }\NormalTok{ToothGrowth}\OperatorTok{$}\NormalTok{len[ToothGrowth}\OperatorTok{$}\NormalTok{supp }\OperatorTok{==}\StringTok{ 'VC'} \OperatorTok{&}\StringTok{ }\NormalTok{ToothGrowth}\OperatorTok{$}\NormalTok{dose }\OperatorTok{==}\StringTok{ }\DecValTok{2}\NormalTok{]}

\KeywordTok{t.test}\NormalTok{(OJ2mg, VC2mg, }\DataTypeTok{alternative =} \StringTok{"two.sided"}\NormalTok{, }\DataTypeTok{paired =} \OtherTok{FALSE}\NormalTok{, }\DataTypeTok{var.equal =} \OtherTok{FALSE}\NormalTok{, }\DataTypeTok{conf.level =} \FloatTok{0.95}\NormalTok{)}
\end{Highlighting}
\end{Shaded}

\begin{verbatim}
## 
##  Welch Two Sample t-test
## 
## data:  OJ2mg and VC2mg
## t = -0.046136, df = 14.04, p-value = 0.9639
## alternative hypothesis: true difference in means is not equal to 0
## 95 percent confidence interval:
##  -3.79807  3.63807
## sample estimates:
## mean of x mean of y 
##     26.06     26.14
\end{verbatim}

Hypothesis to dosage.

The null hypothesis now is that there is not difference in tooth growth
between dosage and our alternative hypotesis is that when bigger the
dosage greater the growth tooth effect.

\begin{Shaded}
\begin{Highlighting}[]
\NormalTok{dose05 =}\StringTok{ }\NormalTok{ToothGrowth}\OperatorTok{$}\NormalTok{len[ToothGrowth}\OperatorTok{$}\NormalTok{dose }\OperatorTok{==}\StringTok{ '0.5'}\NormalTok{]}
\NormalTok{dose10 =}\StringTok{ }\NormalTok{ToothGrowth}\OperatorTok{$}\NormalTok{len[ToothGrowth}\OperatorTok{$}\NormalTok{dose }\OperatorTok{==}\StringTok{ '1'}\NormalTok{]}
\NormalTok{dose20 =}\StringTok{ }\NormalTok{ToothGrowth}\OperatorTok{$}\NormalTok{len[ToothGrowth}\OperatorTok{$}\NormalTok{dose }\OperatorTok{==}\StringTok{ '2'}\NormalTok{]}

\KeywordTok{t.test}\NormalTok{(dose10, dose05, }\DataTypeTok{alternative =} \StringTok{"greater"}\NormalTok{, }\DataTypeTok{paired =} \OtherTok{FALSE}\NormalTok{, }\DataTypeTok{var.equal =} \OtherTok{FALSE}\NormalTok{, }\DataTypeTok{conf.level =} \FloatTok{0.95}\NormalTok{)}
\end{Highlighting}
\end{Shaded}

\begin{verbatim}
## 
##  Welch Two Sample t-test
## 
## data:  dose10 and dose05
## t = 6.4766, df = 37.986, p-value = 6.342e-08
## alternative hypothesis: true difference in means is greater than 0
## 95 percent confidence interval:
##  6.753323      Inf
## sample estimates:
## mean of x mean of y 
##    19.735    10.605
\end{verbatim}

\begin{Shaded}
\begin{Highlighting}[]
\KeywordTok{t.test}\NormalTok{(dose20, dose10, }\DataTypeTok{alternative =} \StringTok{"greater"}\NormalTok{, }\DataTypeTok{paired =} \OtherTok{FALSE}\NormalTok{, }\DataTypeTok{var.equal =} \OtherTok{FALSE}\NormalTok{, }\DataTypeTok{conf.level =} \FloatTok{0.95}\NormalTok{)}
\end{Highlighting}
\end{Shaded}

\begin{verbatim}
## 
##  Welch Two Sample t-test
## 
## data:  dose20 and dose10
## t = 4.9005, df = 37.101, p-value = 9.532e-06
## alternative hypothesis: true difference in means is greater than 0
## 95 percent confidence interval:
##  4.17387     Inf
## sample estimates:
## mean of x mean of y 
##    26.100    19.735
\end{verbatim}

\hypertarget{state-your-conclusions-and-the-assumptions-needed-for-your-conclusions.}{%
\subsection{State your conclusions and the assumptions needed for your
conclusions.}\label{state-your-conclusions-and-the-assumptions-needed-for-your-conclusions.}}

In our t.test, p-values that are lower than 5\% are reject the null
hypothesis. we can conclude that more dosages of vitamin C will result
in bigger tooth growth (p-value\textless{}0.05) and for dosages 0,5 mg
and 1,0 mg the supplement OJ has more effect than VC
(p-value\textless{}0.05. For dosage 2,0 mg results shows that is not
difference between supplements for tooth growth
(p-value\textgreater{}0.05.


\end{document}
