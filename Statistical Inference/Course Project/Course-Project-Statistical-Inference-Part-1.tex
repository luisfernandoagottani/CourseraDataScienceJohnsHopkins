\documentclass[]{article}
\usepackage{lmodern}
\usepackage{amssymb,amsmath}
\usepackage{ifxetex,ifluatex}
\usepackage{fixltx2e} % provides \textsubscript
\ifnum 0\ifxetex 1\fi\ifluatex 1\fi=0 % if pdftex
  \usepackage[T1]{fontenc}
  \usepackage[utf8]{inputenc}
\else % if luatex or xelatex
  \ifxetex
    \usepackage{mathspec}
  \else
    \usepackage{fontspec}
  \fi
  \defaultfontfeatures{Ligatures=TeX,Scale=MatchLowercase}
\fi
% use upquote if available, for straight quotes in verbatim environments
\IfFileExists{upquote.sty}{\usepackage{upquote}}{}
% use microtype if available
\IfFileExists{microtype.sty}{%
\usepackage{microtype}
\UseMicrotypeSet[protrusion]{basicmath} % disable protrusion for tt fonts
}{}
\usepackage[margin=1in]{geometry}
\usepackage{hyperref}
\hypersetup{unicode=true,
            pdftitle={Statistical Inference Course Project - Part 1},
            pdfauthor={Luis Fernando Agottani},
            pdfborder={0 0 0},
            breaklinks=true}
\urlstyle{same}  % don't use monospace font for urls
\usepackage{color}
\usepackage{fancyvrb}
\newcommand{\VerbBar}{|}
\newcommand{\VERB}{\Verb[commandchars=\\\{\}]}
\DefineVerbatimEnvironment{Highlighting}{Verbatim}{commandchars=\\\{\}}
% Add ',fontsize=\small' for more characters per line
\usepackage{framed}
\definecolor{shadecolor}{RGB}{248,248,248}
\newenvironment{Shaded}{\begin{snugshade}}{\end{snugshade}}
\newcommand{\AlertTok}[1]{\textcolor[rgb]{0.94,0.16,0.16}{#1}}
\newcommand{\AnnotationTok}[1]{\textcolor[rgb]{0.56,0.35,0.01}{\textbf{\textit{#1}}}}
\newcommand{\AttributeTok}[1]{\textcolor[rgb]{0.77,0.63,0.00}{#1}}
\newcommand{\BaseNTok}[1]{\textcolor[rgb]{0.00,0.00,0.81}{#1}}
\newcommand{\BuiltInTok}[1]{#1}
\newcommand{\CharTok}[1]{\textcolor[rgb]{0.31,0.60,0.02}{#1}}
\newcommand{\CommentTok}[1]{\textcolor[rgb]{0.56,0.35,0.01}{\textit{#1}}}
\newcommand{\CommentVarTok}[1]{\textcolor[rgb]{0.56,0.35,0.01}{\textbf{\textit{#1}}}}
\newcommand{\ConstantTok}[1]{\textcolor[rgb]{0.00,0.00,0.00}{#1}}
\newcommand{\ControlFlowTok}[1]{\textcolor[rgb]{0.13,0.29,0.53}{\textbf{#1}}}
\newcommand{\DataTypeTok}[1]{\textcolor[rgb]{0.13,0.29,0.53}{#1}}
\newcommand{\DecValTok}[1]{\textcolor[rgb]{0.00,0.00,0.81}{#1}}
\newcommand{\DocumentationTok}[1]{\textcolor[rgb]{0.56,0.35,0.01}{\textbf{\textit{#1}}}}
\newcommand{\ErrorTok}[1]{\textcolor[rgb]{0.64,0.00,0.00}{\textbf{#1}}}
\newcommand{\ExtensionTok}[1]{#1}
\newcommand{\FloatTok}[1]{\textcolor[rgb]{0.00,0.00,0.81}{#1}}
\newcommand{\FunctionTok}[1]{\textcolor[rgb]{0.00,0.00,0.00}{#1}}
\newcommand{\ImportTok}[1]{#1}
\newcommand{\InformationTok}[1]{\textcolor[rgb]{0.56,0.35,0.01}{\textbf{\textit{#1}}}}
\newcommand{\KeywordTok}[1]{\textcolor[rgb]{0.13,0.29,0.53}{\textbf{#1}}}
\newcommand{\NormalTok}[1]{#1}
\newcommand{\OperatorTok}[1]{\textcolor[rgb]{0.81,0.36,0.00}{\textbf{#1}}}
\newcommand{\OtherTok}[1]{\textcolor[rgb]{0.56,0.35,0.01}{#1}}
\newcommand{\PreprocessorTok}[1]{\textcolor[rgb]{0.56,0.35,0.01}{\textit{#1}}}
\newcommand{\RegionMarkerTok}[1]{#1}
\newcommand{\SpecialCharTok}[1]{\textcolor[rgb]{0.00,0.00,0.00}{#1}}
\newcommand{\SpecialStringTok}[1]{\textcolor[rgb]{0.31,0.60,0.02}{#1}}
\newcommand{\StringTok}[1]{\textcolor[rgb]{0.31,0.60,0.02}{#1}}
\newcommand{\VariableTok}[1]{\textcolor[rgb]{0.00,0.00,0.00}{#1}}
\newcommand{\VerbatimStringTok}[1]{\textcolor[rgb]{0.31,0.60,0.02}{#1}}
\newcommand{\WarningTok}[1]{\textcolor[rgb]{0.56,0.35,0.01}{\textbf{\textit{#1}}}}
\usepackage{graphicx,grffile}
\makeatletter
\def\maxwidth{\ifdim\Gin@nat@width>\linewidth\linewidth\else\Gin@nat@width\fi}
\def\maxheight{\ifdim\Gin@nat@height>\textheight\textheight\else\Gin@nat@height\fi}
\makeatother
% Scale images if necessary, so that they will not overflow the page
% margins by default, and it is still possible to overwrite the defaults
% using explicit options in \includegraphics[width, height, ...]{}
\setkeys{Gin}{width=\maxwidth,height=\maxheight,keepaspectratio}
\IfFileExists{parskip.sty}{%
\usepackage{parskip}
}{% else
\setlength{\parindent}{0pt}
\setlength{\parskip}{6pt plus 2pt minus 1pt}
}
\setlength{\emergencystretch}{3em}  % prevent overfull lines
\providecommand{\tightlist}{%
  \setlength{\itemsep}{0pt}\setlength{\parskip}{0pt}}
\setcounter{secnumdepth}{0}
% Redefines (sub)paragraphs to behave more like sections
\ifx\paragraph\undefined\else
\let\oldparagraph\paragraph
\renewcommand{\paragraph}[1]{\oldparagraph{#1}\mbox{}}
\fi
\ifx\subparagraph\undefined\else
\let\oldsubparagraph\subparagraph
\renewcommand{\subparagraph}[1]{\oldsubparagraph{#1}\mbox{}}
\fi

%%% Use protect on footnotes to avoid problems with footnotes in titles
\let\rmarkdownfootnote\footnote%
\def\footnote{\protect\rmarkdownfootnote}

%%% Change title format to be more compact
\usepackage{titling}

% Create subtitle command for use in maketitle
\providecommand{\subtitle}[1]{
  \posttitle{
    \begin{center}\large#1\end{center}
    }
}

\setlength{\droptitle}{-2em}

  \title{Statistical Inference Course Project - Part 1}
    \pretitle{\vspace{\droptitle}\centering\huge}
  \posttitle{\par}
    \author{Luis Fernando Agottani}
    \preauthor{\centering\large\emph}
  \postauthor{\par}
      \predate{\centering\large\emph}
  \postdate{\par}
    \date{24/08/2019}


\begin{document}
\maketitle

The first part of this project has the intention to study how the
Central Limit Theorem works with means from simulations with exponential
distribution.

\hypertarget{part-1---a-simulation-exercise.}{%
\subsection{Part 1 - A simulation
exercise.}\label{part-1---a-simulation-exercise.}}

The goal of the first step is input the mean of each 1000 simulations
with 40 exponentials, where lambda=0.2, rexp(n,lambda) and compare with
de mean of the exponencial distribution, which is 1/lambda.

\begin{Shaded}
\begin{Highlighting}[]
\NormalTok{n=}\DecValTok{40}
\NormalTok{simulations=}\DecValTok{1000}
\NormalTok{lambda=}\FloatTok{0.2}

\NormalTok{mns =}\StringTok{ }\OtherTok{NULL}
\ControlFlowTok{for}\NormalTok{ (i }\ControlFlowTok{in} \DecValTok{1} \OperatorTok{:}\StringTok{ }\DecValTok{1000}\NormalTok{) mns =}\StringTok{ }\KeywordTok{c}\NormalTok{(mns,}
                            \KeywordTok{mean}\NormalTok{(}\KeywordTok{rexp}\NormalTok{(n,lambda)))}
\end{Highlighting}
\end{Shaded}

\begin{enumerate}
\def\labelenumi{\arabic{enumi}.}
\tightlist
\item
  Show the sample mean and compare it to the theoretical mean of the
  distribution.
\end{enumerate}

\begin{Shaded}
\begin{Highlighting}[]
\NormalTok{Theoricalmean<-}\StringTok{ }\DecValTok{1}\OperatorTok{/}\NormalTok{lambda}

\NormalTok{Theoricalmean}
\end{Highlighting}
\end{Shaded}

\begin{verbatim}
## [1] 5
\end{verbatim}

\begin{Shaded}
\begin{Highlighting}[]
\NormalTok{Samplemean<-}\StringTok{ }\KeywordTok{mean}\NormalTok{(mns)}

\NormalTok{Samplemean}
\end{Highlighting}
\end{Shaded}

\begin{verbatim}
## [1] 5.036567
\end{verbatim}

\begin{enumerate}
\def\labelenumi{\arabic{enumi}.}
\setcounter{enumi}{1}
\tightlist
\item
  Show how variable the sample is (via variance) and compare it to the
  theoretical variance of the distribution.
\end{enumerate}

\begin{Shaded}
\begin{Highlighting}[]
\NormalTok{Theoricalvariance<-}\StringTok{ }\NormalTok{(}\DecValTok{1}\OperatorTok{/}\NormalTok{lambda}\OperatorTok{/}\KeywordTok{sqrt}\NormalTok{(}\DecValTok{40}\NormalTok{))}\OperatorTok{^}\DecValTok{2}

\NormalTok{Theoricalvariance}
\end{Highlighting}
\end{Shaded}

\begin{verbatim}
## [1] 0.625
\end{verbatim}

\begin{Shaded}
\begin{Highlighting}[]
\NormalTok{Samplevariance<-}\StringTok{ }\KeywordTok{var}\NormalTok{(mns)}

\NormalTok{Samplevariance}
\end{Highlighting}
\end{Shaded}

\begin{verbatim}
## [1] 0.5758694
\end{verbatim}

As we can see, the sample mean and variance are very similar to the
therorical.

\begin{enumerate}
\def\labelenumi{\arabic{enumi}.}
\setcounter{enumi}{2}
\tightlist
\item
  Show that the distribution is approximately normal.
\end{enumerate}

\begin{Shaded}
\begin{Highlighting}[]
\KeywordTok{hist}\NormalTok{(mns, }
     \DataTypeTok{breaks=} \DecValTok{25}\NormalTok{,}
     \DataTypeTok{prob =} \OtherTok{TRUE}\NormalTok{,}
     \DataTypeTok{col=}\StringTok{"grey"}\NormalTok{, }
     \DataTypeTok{border=}\StringTok{"white"}\NormalTok{, }
     \DataTypeTok{xlab=}\StringTok{"Means"}\NormalTok{, }
     \DataTypeTok{main=}\StringTok{"Means distribution from exponencial simulation"}\NormalTok{)}
  
\KeywordTok{abline}\NormalTok{(}\DataTypeTok{v=}\KeywordTok{mean}\NormalTok{(mns), }
       \DataTypeTok{col=}\StringTok{"black"}\NormalTok{,}
       \DataTypeTok{lwd=}\DecValTok{6}\NormalTok{)}

\KeywordTok{abline}\NormalTok{(}\DataTypeTok{v=}\DecValTok{5}\NormalTok{, }
       \DataTypeTok{col=}\StringTok{"red"}\NormalTok{,}
       \DataTypeTok{lwd=}\DecValTok{4}\NormalTok{)}

\KeywordTok{legend}\NormalTok{(}\DataTypeTok{x=}\StringTok{"topright"}\NormalTok{,}
      \KeywordTok{c}\NormalTok{(}\StringTok{"Sample Mean"}\NormalTok{,}\StringTok{"Theorical Mean"}\NormalTok{, }\StringTok{"Theorical Distribution"}\NormalTok{),}
      \DataTypeTok{col=}\KeywordTok{c}\NormalTok{(}\StringTok{"black"}\NormalTok{,}\StringTok{"red"}\NormalTok{, }\StringTok{"blue"}\NormalTok{),}
      \DataTypeTok{lty=}\KeywordTok{c}\NormalTok{(}\DecValTok{1}\NormalTok{,}\DecValTok{1}\NormalTok{))}

\NormalTok{x <-}\StringTok{ }\KeywordTok{seq}\NormalTok{(}\KeywordTok{min}\NormalTok{(mns), }\KeywordTok{max}\NormalTok{(mns), }\DataTypeTok{length =} \DecValTok{100}\NormalTok{)}

\KeywordTok{lines}\NormalTok{(x, }\KeywordTok{dnorm}\NormalTok{(x, }\DataTypeTok{mean =} \DecValTok{1}\OperatorTok{/}\NormalTok{lambda, }\DataTypeTok{sd =}\NormalTok{ (}\DecValTok{1}\OperatorTok{/}\NormalTok{lambda}\OperatorTok{/}\KeywordTok{sqrt}\NormalTok{(n))), }\DataTypeTok{pch =} \DecValTok{25}\NormalTok{, }\DataTypeTok{col =} \StringTok{"blue"}\NormalTok{)}
\end{Highlighting}
\end{Shaded}

\includegraphics{Course-Project-Statistical-Inference-Part-1_files/figure-latex/plotNORMAL-1.pdf}

So, the blue line represents the theorical distribuition, which is
similar to the sample mean.


\end{document}
